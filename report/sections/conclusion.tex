\documentclass[../report.tex]{subfiles}

\begin{document}

As stated, the goal of the project was to create a Physically Based Renderer, and while we were ultimately unable to reach this goal, we did succeed in creating a 3D renderer in Vulkan, a graphics API that is notoriously difficult to learn and use, which is a goal in and of itself. Moreover, while the output may not be the most visually impressive, it still took an inordinate amount of complexity to get to the point we got to.

We decided it would be better to implement a raster engine instead, because we already had most of the setup done for Vulkan, and at that point, continuing to attempt to implement PBR would have pushed us over our time constraints, moreover the theory behind rasterisation is much simpler and easier to grasp than PBR and there are much more resources out there as a result.

However, just because we did not make a physically based renderer in time, does not mean that we won't be able to in the future, in fact the application we created is actually a very strong foundation for implementing one in the future, or we could focus further on rasterisation and use it to develop a game engine, capable of rendering entire scenes in fractions of a second.
Although it would take months or even years, the application as it currently exists could be expanded and extended with more and more advanced features over time.

Because Vulkan requires the implemeters to implement so much of the core rendering system, there is a much greater degree of potential optimisation to be found, and more and more large game companies are using Vulkan to make their games, so experience working with Vulkan has the potential to be extremely desired in the near future.

\end{document}


