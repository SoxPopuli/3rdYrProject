\documentclass[../report.tex]{subfiles}

\begin{document}

%  \definecolor{tablerow1}{gray}{1.0}
%  \definecolor{tablerow2}{gray}{0.8}
% 
%  \rowcolors{1}{tablerow1}{tablerow2}
% \begin{center}
% 	\begin{tabular}{ | m{.3\textwidth} | m{.5\textwidth} | }
% 		\hline
% 		
%         \textbf{Rendering} & {Producing a 2D image from virtual scene input} \\
% 		\textbf{Physically Based Rendering} & test description \\
% 		\textbf{test 3} & test description \\
% 		\textbf{test 4} & test description \\
% 		\textbf{test 5} & test description \\
% 		
% 		\hline
% 	\end{tabular}
% \end{center}

\begin{itemize}
    \item \textbf{Rendering}: Producing a 2D image from virtual 3D scene input 
    \item \textbf{Physically Based Rendering}: Rendering using techniques that model the flow of light in the real world.
    \item \textbf{Pipeline}: The process that the various inputs go through to produce an output.

    \item \textbf{Bindings}: Also called glue code, it is code that allows a language to call code written in another language.
    \item \textbf{Compilation}: converting textual source code into binary machine code
    \item \textbf{Cross-Compilation}: The act of compiling code for a different system or architecture than the one that compiled the code.
    \item \textbf{Windowing}: Processes involved with creating and handling application windows, usually at the OS level
    \item \textbf{Swapchain}: an array of images the GPU swaps between in order to display images on the screen.
    \item \textbf{Occlusion}: When an object is blocked by another object.
\end{itemize}

\end{document}
